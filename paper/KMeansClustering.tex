\documentclass[10pt,twocolumn,letterpaper]{article}

\usepackage{cvpr}
\usepackage{times}
\usepackage{epsfig}
\usepackage{graphicx}
\usepackage{amsmath}
\usepackage{amssymb}

\usepackage{url}

% Include other packages here, before hyperref.
\usepackage{algorithm}
\usepackage{algpseudocode}

% If you comment hyperref and then uncomment it, you should delete
% egpaper.aux before re-running latex.  (Or just hit 'q' on the first latex
% run, let it finish, and you should be clear).
%\usepackage[pagebackref=true,breaklinks=true,letterpaper=true,colorlinks,bookmarks=false]{hyperref}

\cvprfinalcopy % *** Uncomment this line for the final submission

\def\cvprPaperID{****} % *** Enter the CVPR Paper ID here
\def\httilde{\mbox{\tt\raisebox{-.5ex}{\symbol{126}}}}

% Pages are numbered in submission mode, and unnumbered in camera-ready
\ifcvprfinal\pagestyle{empty}\fi
\begin{document}

%%%%%%%%% TITLE
\title{K-Means Clustering: comparison between a sequential and parallel implementation}

\author{Beatrice Paoli\\
{\tt\small beatrice.paoli@stud.unifi.it}
}

\maketitle
\thispagestyle{empty}

%%%%%%%%% ABSTRACT
\begin{abstract}
	The goal of this paper is to present two implementations of the k-means clustering algorithm: a sequential version and a parallel one. Both are written in C++, with the parallel version obtained with the use of OpenMP. This paper also provides a performance analysis and comparison between the two versions in terms of execution times and speedup and how this values change with different inputs (dataset size and number of centers) and different numbers of threads. 
\end{abstract}

%-------------------------------------------------------------------------

\section{The Algorithm}

The K-means algorithm is a parametrized clustering technique that allows to partition a dataset of points or observations into K clusters. Points in the same cluster minimize the distance from the center of the cluster (called \textit{prototype}). 
Many metrics of distances can be used depending on the dataset to cluster, for this implementation the dataset is comprised of 2D points and distances are measured with the Euclidean distance.

The algorithm is composed by few steps, with two main phases alternated between each other: the \textit{Assignment} phase and the \textit{Update} phase. %\ref{alg:KMeansAlg}

\begin{algorithm}
	\label{alg:KMeansAlg}
	\caption{K-Means Clustering}
	\begin{algorithmic}
		\Require K = number of clusters to create
		\vspace{0.5cm}
		
		\State Select K points as initial centroids of the clusters
		\While{Centroids keep changing}
		\For{each point $p$}
		\State Assign $p$ to the cluster with the closest centroid
		\EndFor
		\For{each cluster $c$}
		\State Update the centroid of $c$
		\EndFor
		\EndWhile
		\State \Return clusters
		
	\end{algorithmic}
\end{algorithm}

The algorithm converges after few steps to a local minimum, depending on the choice of the initial centroids. In the implementation presented in this paper, the starting centroids are chosen randomly from the dataset in input. Therefor, different runs of the algorithm on the same dataset and the same number of centroids can yield different results. Other initialization methods can be used to obtain the gloabl optimum more consistently.

%-------------------------------------------------------------------------

\section{Implementation}
The details of the implementations of the function \textit{kMeansClustering()} are presented in the following paragraphs.

\subsection{Classes}
\begin{itemize}
	\item \textbf{Point}: this class is used to represent the points of the input dataset to cluster. It has three members: \textit{x} and \textit{y} for the coordinates, and \textit{clusterId} for the id of the cluster to which the point has been assigned. 
	The class also has two constructors and the method \textit{dist()} to compute the Euclidean distance between two points.
	\item \textbf{Cluster}: this class is used to represent the clusters created by the algorithm. Each cluster has an \textit{id} ranging from 0 to $K - 1$, a vector of objects of type \textit{Point} containing the points assigned to the cluster, a \textit{Point} for the current mean or centroid of the cluster, and two fields to compute the partial sums of all points for each coordinate, \textit{tempSumX} and \textit{tempSumY}.
	Aside from the constructor, the class has two main methods:
	\begin{itemize}
		\item \textit{addPoint()}: is used during the Assignment phase; it adds a point to the dataset and adds its coordinates to the partial sums.
		\item \textit{updateCentroid()}: is used during the Update phase; it computes the new mean of the cluster by using the sums of the coordinates computed before and the size of the list of points assigned. After the update, \textit{tempSumX} and \textit{tempSumY} are resetted to 0, ready for a new iteration of the algorithm.
	\end{itemize}
\end{itemize}

\subsection{kMeansClustering}

%-------------------------------------------------------------------------
\section{Roba da togliere}

\subsection{Miscellaneous}

\noindent
Compare the following:\\
\begin{tabular}{ll}
 \verb'$conf_a$' &  $conf_a$ \\
 \verb'$\mathit{conf}_a$' & $\mathit{conf}_a$
\end{tabular}\\
See The \TeX book, p165.

The space after \eg, meaning ``for example'', should not be a
sentence-ending space. So \eg is correct, {\em e.g.} is not.  The provided
\verb'\eg' macro takes care of this.

When citing a multi-author paper, you may save space by using ``et alia'',
shortened to ``\etal'' (not ``{\em et.\ al.}'' as ``{\em et}'' is a complete word.)
However, use it only when there are three or more authors.  Thus, the
following is correct: ``
   Frobnication has been trendy lately.
   It was introduced by Alpher~\cite{Alpher02}, and subsequently developed by
   Alpher and Fotheringham-Smythe~\cite{Alpher03}, and Alpher \etal~\cite{Alpher04}.''

This is incorrect: ``... subsequently developed by Alpher \etal~\cite{Alpher03} ...''
because reference~\cite{Alpher03} has just two authors.  If you use the
\verb'\etal' macro provided, then you need not worry about double periods
when used at the end of a sentence as in Alpher \etal.

For this citation style, keep multiple citations in numerical (not
chronological) order, so prefer \cite{Alpher03,Alpher02,Authors06} to
\cite{Alpher02,Alpher03,Authors06}.


\begin{figure*}
\begin{center}
\fbox{\rule{0pt}{2in} \rule{.9\linewidth}{0pt}}
\end{center}
   \caption{Example of a short caption, which should be centered.}
\label{fig:short}
\end{figure*}

%------------------------------------------------------------------------
\section{Formatting your paper}

All text must be in a two-column format. The total allowable width of the
text area is $6\frac78$ inches (17.5 cm) wide by $8\frac78$ inches (22.54
cm) high. Columns are to be $3\frac14$ inches (8.25 cm) wide, with a
$\frac{5}{16}$ inch (0.8 cm) space between them. The main title (on the
first page) should begin 1.0 inch (2.54 cm) from the top edge of the
page. The second and following pages should begin 1.0 inch (2.54 cm) from
the top edge. On all pages, the bottom margin should be 1-1/8 inches (2.86
cm) from the bottom edge of the page for $8.5 \times 11$-inch paper; for A4
paper, approximately 1-5/8 inches (4.13 cm) from the bottom edge of the
page.

%-------------------------------------------------------------------------
\subsection{Margins and page numbering}

All printed material, including text, illustrations, and charts, must be
kept within a print area 6-7/8 inches (17.5 cm) wide by 8-7/8 inches
(22.54 cm) high.


%-------------------------------------------------------------------------
\subsection{Type-style and fonts}

Wherever Times is specified, Times Roman may also be used. If neither is
available on your word processor, please use the font closest in
appearance to Times to which you have access.

MAIN TITLE. Center the title 1-3/8 inches (3.49 cm) from the top edge of
the first page. The title should be in Times 14-point, boldface type.
Capitalize the first letter of nouns, pronouns, verbs, adjectives, and
adverbs; do not capitalize articles, coordinate conjunctions, or
prepositions (unless the title begins with such a word). Leave two blank
lines after the title.

AUTHOR NAME(s) and AFFILIATION(s) are to be centered beneath the title
and printed in Times 12-point, non-boldface type. This information is to
be followed by two blank lines.

The ABSTRACT and MAIN TEXT are to be in a two-column format.

MAIN TEXT. Type main text in 10-point Times, single-spaced. Do NOT use
double-spacing. All paragraphs should be indented 1 pica (approx. 1/6
inch or 0.422 cm). Make sure your text is fully justified---that is,
flush left and flush right. Please do not place any additional blank
lines between paragraphs.

Figure and table captions should be 9-point Roman type as in
Figures~\ref{fig:onecol} and~\ref{fig:short}.  Short captions should be centred.

\noindent Callouts should be 9-point Helvetica, non-boldface type.
Initially capitalize only the first word of section titles and first-,
second-, and third-order headings.

FIRST-ORDER HEADINGS. (For example, {\large \bf 1. Introduction})
should be Times 12-point boldface, initially capitalized, flush left,
with one blank line before, and one blank line after.

SECOND-ORDER HEADINGS. (For example, { \bf 1.1. Database elements})
should be Times 11-point boldface, initially capitalized, flush left,
with one blank line before, and one after. If you require a third-order
heading (we discourage it), use 10-point Times, boldface, initially
capitalized, flush left, preceded by one blank line, followed by a period
and your text on the same line.

%-------------------------------------------------------------------------
\subsection{Footnotes}

Please use footnotes\footnote {This is what a footnote looks like.  It
often distracts the reader from the main flow of the argument.} sparingly.
Indeed, try to avoid footnotes altogether and include necessary peripheral
observations in
the text (within parentheses, if you prefer, as in this sentence).  If you
wish to use a footnote, place it at the bottom of the column on the page on
which it is referenced. Use Times 8-point type, single-spaced.


%-------------------------------------------------------------------------
\subsection{References}

List and number all bibliographical references in 9-point Times,
single-spaced, at the end of your paper. When referenced in the text,
enclose the citation number in square brackets, for
example~\cite{Authors06}.  Where appropriate, include the name(s) of
editors of referenced books.

\begin{table}
\begin{center}
\begin{tabular}{|l|c|}
\hline
Method & Frobnability \\
\hline\hline
Theirs & Frumpy \\
Yours & Frobbly \\
Ours & Makes one's heart Frob\\
\hline
\end{tabular}
\end{center}
\caption{Results.   Ours is better.}
\end{table}

%-------------------------------------------------------------------------
\subsection{Illustrations, graphs, and photographs}

All graphics should be centered.  Please ensure that any point you wish to
make is resolvable in a printed copy of the paper.  Resize fonts in figures
to match the font in the body text, and choose line widths which render
effectively in print.  Many readers (and reviewers), even of an electronic
copy, will choose to print your paper in order to read it.  You cannot
insist that they do otherwise, and therefore must not assume that they can
zoom in to see tiny details on a graphic.

When placing figures in \LaTeX, it's almost always best to use
\verb+\includegraphics+, and to specify the  figure width as a multiple of
the line width as in the example below
{\small\begin{verbatim}
   \usepackage[dvips]{graphicx} ...
   \includegraphics[width=0.8\linewidth]
                   {myfile.eps}
\end{verbatim}
}


%-------------------------------------------------------------------------
\subsection{Color}

Color is valuable, and will be visible to readers of the electronic copy.
However ensure that, when printed on a monochrome printer, no important
information is lost by the conversion to grayscale.

{\small
\bibliographystyle{ieee}
\bibliography{egbib}
}

\section{Appendix}
If your course project is part of a larger project from another class or research lab, please fill in this section and clearly spell out the following items:

\begin{enumerate}
\item  Explicitly explain what the computer vision components are in this course project;
\item  Explicitly list out all of your own contributions in this project in terms of:
	\begin{enumerate}
	\item ideas
	\item formulations of algorithms
 	\item software and coding
	\item designs of experiments
	\item analysis of experiments
	\end{enumerate}
\item Verify and confirm that you (and your partner currently taking CS231A) are the sole author(s) of the writeup.
Please provide papers, theses, or other documents related to this project so that we can compare with your own writeup.
\end{enumerate}
\end{document}
