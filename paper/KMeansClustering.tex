\documentclass[10pt,twocolumn,letterpaper]{article}

\usepackage{cvpr}
\usepackage{times}
\usepackage{epsfig}
\usepackage{graphicx}
\usepackage{amsmath}
\usepackage{amssymb}

\usepackage{url}

% Include other packages here, before hyperref.
\usepackage{algorithm}
\usepackage{algpseudocode}
\usepackage{xcolor}

% If you comment hyperref and then uncomment it, you should delete
% egpaper.aux before re-running latex.  (Or just hit 'q' on the first latex
% run, let it finish, and you should be clear).
%\usepackage[pagebackref=true,breaklinks=true,letterpaper=true,colorlinks,bookmarks=false]{hyperref}

\cvprfinalcopy % *** Uncomment this line for the final submission

\def\cvprPaperID{****} % *** Enter the CVPR Paper ID here
\def\httilde{\mbox{\tt\raisebox{-.5ex}{\symbol{126}}}}

% Pages are numbered in submission mode, and unnumbered in camera-ready
\ifcvprfinal\pagestyle{empty}\fi
\begin{document}

%%%%%%%%% TITLE
\title{K-Means Clustering: comparison between a sequential and parallel implementation}

\author{Beatrice Paoli\\
{\tt\small beatrice.paoli@stud.unifi.it}
}

\maketitle
\thispagestyle{empty}

%%%%%%%%% ABSTRACT
\begin{abstract}
	The goal of this paper is to present two implementations of the k-means clustering algorithm: a sequential version and a parallel one. Both are written in C++, with the parallel version obtained with the use of OpenMP. This paper also provides a performance analysis and comparison between the two versions in terms of execution times and speedup and how this values change with different inputs (dataset size and number of centers) and different numbers of threads. 
\end{abstract}

%-------------------------------------------------------------------------

\section{The Algorithm}

The K-means algorithm is a parametrized clustering technique that allows to partition a dataset of points or observations into K clusters. Points in the same cluster minimize the distance from the center of the cluster (called \textit{prototype}). 

Many metrics of distances can be used depending on the dataset to cluster, for this implementation the dataset is comprised of 2D points and distances are measured with the Euclidean distance.

The algorithm is composed by few steps, with two main phases alternated between each other: the \textit{Assignment} phase and the \textit{Update} phase. %\ref{alg:KMeansAlg}

\begin{algorithm}
	\label{alg:KMeansAlg}
	\caption{K-Means Clustering}
	\begin{algorithmic}
		\Require K = number of clusters to create
		\vspace{0.5cm}
		
		\State Select K points as initial centroids of the clusters
		\While{Centroids keep changing}
		\For{each point $p$}
		\State Assign $p$ to the cluster with the closest centroid
		\EndFor
		\For{each cluster $c$}
		\State Update the centroid of $c$
		\EndFor
		\EndWhile
		\State \Return clusters
		
	\end{algorithmic}
\end{algorithm}

The algorithm converges after few steps to a local minimum, depending on the choice of the initial centroids. In the implementation presented in this paper, the starting centroids are chosen randomly from the dataset in input. Therefor, different runs of the algorithm on the same dataset and the same number of centroids can yield different results. Other initialization methods can be used to obtain the gloabl optimum more consistently.

%-------------------------------------------------------------------------

\section{Implementation}
The details of the implementations of the function \textit{kMeansClustering()} are presented in the following paragraphs.

\subsection{Classes}
\begin{itemize}
	\item \textbf{Point}: this class is used to represent the points of the input dataset to cluster. It has three members: \textit{x} and \textit{y} for the coordinates, and \textit{clusterId} for the id of the cluster to which the point has been assigned. 
	
	The class also has two constructors and the method \textit{dist()} to compute the Euclidean distance between two points.
	
	\item \textbf{Cluster}: this class is used to represent the clusters created by the algorithm. Each cluster has an \textit{id} ranging from 0 to $K - 1$, a \textit{Point} for the current mean or centroid of the cluster, two fields to compute the partial sums of all points for each coordinate, \textit{tempSumX} and \textit{tempSumY}, and an integer for the size of the cluster.
	
	Aside from the constructor, the class has two main methods:
	\begin{itemize}
		\item \textit{addPoint()}: is used during the Assignment phase; it adds a point to the cluster by increasing the \textit{size} counter and by adding its coordinates to the partial sums.
		\item \textit{updateCentroid()}: is used during the Update phase; it computes the new mean of the cluster by using the sums of the coordinates computed before and the size of the cluster. After the update, \textit{tempSumX}, \textit{tempSumY} and \textit{size} are resetted to 0, ready for a new iteration of the algorithm.
	\end{itemize}
\end{itemize}

\subsection{kMeansClustering}
The function \textit{kMeansClustering()} accepts three parameters: an integer \textit{k} for the number of clusters to output, a \textit{std::vector} of type \textit{Point} for the dataset, and an integer for the number of iterations of the algorithm to perform. The last parameter is optional and has a default value of 20.

The function starts by performing a random initialization of the clusters by choosing $k$ points as initial centroids.

In a \textit{for} cycle bounded by the number of iterations to perform, the function alternates the assignment and update steps.
In the assignment step, for each point we find the cluster with the closest centroid and then update the point's \textit{clusterId} and call the cluster's method \textit{addPoint()} to update the partial sums of the future centroid.

The update step is limited to a simple \textit{for} cycle over the clusters to call \textit{updateCetroid()} to update all the centroids with the current points assignment. 

\subsection{OpenMP Parallelization}
To achieve parallelization it was necessary to use only a few parallel directives.

The assignment step is embarrassingly parallel since the search and assignment of each point to a cluster is independent from one another.
So a \verb"#pragma omp for" statement was used on the points loop to use threads to parallelize the computation.

\begin{algorithm}
	\label{alg:KMeansAlgParallel}
	\caption{Parallel K-Means Clustering}
	\begin{algorithmic}
		\Require K = number of clusters to create
		\vspace{0.5cm}
		
		\State Select K points as initial centroids of the clusters
		\While{Centroids keep changing}
		\State \textcolor{orange}{\#pragma omp for}
		\For{each point $p$}
		\State Assign $p$ to the cluster with the closest centroid
		\EndFor
		\For{each cluster $c$}
		\State Update the centroid of $c$
		\EndFor
		\EndWhile
		\State \Return clusters
		
	\end{algorithmic}
\end{algorithm}

Since the workload for each thread is similar the default static scheduler is used.

However, each thread accesses the shared clusters to call \textit{addPoint()} and update the internal cluster variable. Without any synchronization mechanism, this introduces a race condition on the algorithm that leads to incorrect results.

The method \textit{addPoint()} only performs three sums over the variables \textit{tempSumX}, \textit{tempSumY} and \textit{size}, so three \verb"#pragma omp atomic" were used to ensure the correct update of the variables and avoid the race condition.

%-------------------------------------------------------------------------
%\section{Roba da togliere}
%
%\subsection{Miscellaneous}
%
%\noindent
%Compare the following:\\
%\begin{tabular}{ll}
% \verb'$conf_a$' &  $conf_a$ \\
% \verb'$\mathit{conf}_a$' & $\mathit{conf}_a$
%\end{tabular}\\
%See The \TeX book, p165.
%
%The space after \eg, meaning ``for example'', should not be a
%sentence-ending space. So \eg is correct, {\em e.g.} is not.  The provided
%\verb'\eg' macro takes care of this.
%
%When citing a multi-author paper, you may save space by using ``et alia'',
%shortened to ``\etal'' (not ``{\em et.\ al.}'' as ``{\em et}'' is a complete word.)
%However, use it only when there are three or more authors.  Thus, the
%following is correct: ``
%   Frobnication has been trendy lately.
%   It was introduced by Alpher~\cite{Alpher02}, and subsequently developed by
%   Alpher and Fotheringham-Smythe~\cite{Alpher03}, and Alpher \etal~\cite{Alpher04}.''
%
%This is incorrect: ``... subsequently developed by Alpher \etal~\cite{Alpher03} ...''
%because reference~\cite{Alpher03} has just two authors.  If you use the
%\verb'\etal' macro provided, then you need not worry about double periods
%when used at the end of a sentence as in Alpher \etal.
%
%For this citation style, keep multiple citations in numerical (not
%chronological) order, so prefer \cite{Alpher03,Alpher02,Authors06} to
%\cite{Alpher02,Alpher03,Authors06}.
%
%
%\begin{figure*}
%\begin{center}
%\fbox{\rule{0pt}{2in} \rule{.9\linewidth}{0pt}}
%\end{center}
%   \caption{Example of a short caption, which should be centered.}
%\label{fig:short}
%\end{figure*}
%
%
%%-------------------------------------------------------------------------
%\subsection{References}
%
%List and number all bibliographical references in 9-point Times,
%single-spaced, at the end of your paper. When referenced in the text,
%enclose the citation number in square brackets, for
%example~\cite{Authors06}.  Where appropriate, include the name(s) of
%editors of referenced books.
%
%\begin{table}
%\begin{center}
%\begin{tabular}{|l|c|}
%\hline
%Method & Frobnability \\
%\hline\hline
%Theirs & Frumpy \\
%Yours & Frobbly \\
%Ours & Makes one's heart Frob\\
%\hline
%\end{tabular}
%\end{center}
%\caption{Results.   Ours is better.}
%\end{table}
%
%%-------------------------------------------------------------------------
%\subsection{Illustrations, graphs, and photographs}
%
%All graphics should be centered.  Please ensure that any point you wish to
%make is resolvable in a printed copy of the paper.  Resize fonts in figures
%to match the font in the body text, and choose line widths which render
%effectively in print.  Many readers (and reviewers), even of an electronic
%copy, will choose to print your paper in order to read it.  You cannot
%insist that they do otherwise, and therefore must not assume that they can
%zoom in to see tiny details on a graphic.
%
%When placing figures in \LaTeX, it's almost always best to use
%\verb+\includegraphics+, and to specify the  figure width as a multiple of
%the line width as in the example below
%{\small\begin{verbatim}
%   \usepackage[dvips]{graphicx} ...
%   \includegraphics[width=0.8\linewidth]
%                   {myfile.eps}
%\end{verbatim}
%}
%
%\section{Appendix}
%If your course project is part of a larger project from another class or research lab, please fill in this section and clearly spell out the following items:
%
%\begin{enumerate}
%\item  Explicitly explain what the computer vision components are in this course project;
%\item  Explicitly list out all of your own contributions in this project in terms of:
%	\begin{enumerate}
%	\item ideas
%	\item formulations of algorithms
% 	\item software and coding
%	\item designs of experiments
%	\item analysis of experiments
%	\end{enumerate}
%\item Verify and confirm that you (and your partner currently taking CS231A) are the sole author(s) of the writeup.
%Please provide papers, theses, or other documents related to this project so that we can compare with your own writeup.
%\end{enumerate}
\end{document}
